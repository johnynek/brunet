\documentclass[11pt]{article}

\usepackage{graphicx}
\usepackage{fullpage}

\begin{document}
\title{Practical issues about the Brunet Ad-hoc Network}
\author{P. Oscar Boykin}
\date{April 16, 2003}
\maketitle
\begin{abstract}
This is an informal document which explains the hows and whys of the Brunet
system, without going into the details of the specification.  It is meant to
answer questions about properties which may not be clear, or are merely
implied by the protocol itself.  This document also answers questions about
applications of the Brunet system. 
\end{abstract}

\section{Introduction}
The Brunet protocol specifies how nodes may join the brunet network, and how
packets are routed to the appropriate destination nodes.  The routing
algorithms and methods for joining and leaving the network are rather simple,
however, the specification does not suggest any specific applications of the
system.  Each packet has a one byte identifier which nodes can use to
determine the type of data in the payload, however, the routing of packets is
independent of the content.  There a few practical aspects that complicate
the implementation of any theoretically derived protocol.  In particular, many
networks protocols often have low practical limits on the number of
connections (e.g. TCP on many systems), or cannot guarantee the delivery of
messages (e.g. UDP, ethernet).  Additionally, many nodes may only be able to
initiate connections and cannot accept incoming connections.

\section{Firewalls}
Nodes which cannot accept incoming connections are called firewalled nodes.
Firewalled nodes, in general, do not have methods for applications to learn of
the existence or properties of the firewall.  As such, it is desirable to
design a protocol whose rules are uniform for each node, and do not depend on
the knowledge as to whether or not that node is behind a firewall.
\begin{figure}
\centering
\includegraphics[width=3.5in]{example}
\caption{Example of a few nodes with structured edges.  Grey nodes are
firewalled}
\label{fig:firewall}
\end{figure}
In figure \ref{fig:firewall}, we give an example of a section of a network.
In the figure, there are four nodes which are not firewalled (N1,N5,N10,N20),
and there are three that are firewalled (N7,N8,N9).  Each node has two
outgoing connections (those of N1 and N20 are omitted).

There are three questions regarding firewalls that must be addressed: first,
will the routing algorithm insure that each packet will reach its destination,
second, will new nodes joining the network find the proper place, and last,
when nodes exit will the structure remain correct?  We will first answer these
questions for particular examples, and later address the issue with a proof
that it will behave correctly in general.

\subsection{Routing and Firewalled Nodes}
As to the first point, it is clear that all the non-firewalled nodes will be
found by the normal algorithm.  For firewalled nodes, in our example N7,N8 and
N9, the story is not as clear.  Clearly if any node less than or equal to 5,
or greater than or equal to 10 sends a message, it will reach them.  That is
clear because that message will first reach N5 or N10 depending on the
direction, and then it will reach the destination on the next hop.  There is
one last case to consider, suppose N8 wants to address N9.  Recall that the
routing algorithm is the following: always send the packet towards the address
which brings it closest to the destination and decreases total distance
\emph{except when there is no node
closer} in which case you send it to the node in between which the destination
node must be.  Hence, when addressing N9, N8 must send the packet to N10, as
N10 is closer to N9 than N5 is.  Finally, N10 is connected directly to N9, so
it passes the node to N9.  The key to seeing that this routing will work is
that between any two non-firewalled nodes, all the nodes whose addresses lie
in that interval are connected to both end points.  In between those two
non-firewalled nodes, the ordering of the network is not strictly correct, in
that those firewalled nodes are not connected to their nearest neighbors (who
may be themselves firewalled), however, the end points are connected to
\emph{all} the nodes in between, so the extra final step in the routing
algorithm will always make sure a packet can reach its destination.

A side effect of the above structure is that firewalled nodes will never be
used in local routing, but for long range routing they may be recieve
incoming packets on their shortcut connections.

\subsection{Joining the Network}
The second question is: will nodes join in the proper place?  To join, a node
connects to a random node on the network, which acts as a proxy for a
``bootstrap" message.  The bootstrap message is sent on the network to the
address of the joining node.  Lets discuss two examples.  First, consider N3
wants to join the network.  The message will be sent to N3, which will wind up
being delivered to both N1 and N5, as they are the two closest to where N3
should be.  Both N1 and N5 will respond to the request and will connect to N3
if N3 is not firewalled.  If N3 is firewalled, then N3 will connect to both N1
and N5.  In the firewalled case, N3 has two outbound connections, and as such
is done connecting.  In the non-firewalled case, N3 will have two incoming
connections, one from N1 and one from N5.  Now, it still have to create two
outbound connections to the nearest neighbors not already connected to.  On
connection, N3 will learn the nearest nieghbors of N1 and N5.  N3 may then use
this information to find the nodes nearest to it which are not firewalled, to
which it may connect.  Since N1 must have at least one outbound connection to
the left, there exists at least one near neighbor for N3 to connect to towards
the left.  Likewise for N5.

The more complicated cases involve firewalled having addresses nearest to the
joining node.  In particular, there are two cases: one of the nearest
neighbors is firewalled or both of the nearest neighbors is firewalled.
Suppose that N6 wants to join the network.  Clearly N5 and N7 (being the two
nearest nodes, and neighbors of one another) will recieve the bootstrap
message.  If N6 is not firewalled, then N5 and N7 will connect to N6.  After
N6 is connected to N5 and N7, both of the latter nodes will communicate that
each of their neighbors, some of whom may connect to N6.  In the case that N6
is firewalled, N6 will be able to connect to N5, but N6 and N7 will not be
able to connect.  After connecting to N5, N6 will learn the neighbors of N5,
and as such will be able to connect to the proper nodes.  If we consider the
node N8.5, a node that should be situated between N8 and N9, we see a slightly
more complicated picture.  If the packet comes by way of N10, the routing
rules say that the packet should then be sent to N8\footnote{the routing
rule says to route to the node further from the current node when there are two
equally distant from the destination}.  The node N8 will see that it cannot
route the packet to a node nearer to N8.5, and will route to the second
nearest node (on the other side of 8.5), which is back to N10.
When N10 recieves the packet from a node which is closer to the destination,
and it cannot route the packet directly to N8.5, then the packet stops at this
point and N10 may process it.
Thus, N8.5 will join
between N8 and N10.  Once connected to N10, it will also find the addresses of
N9 and N5.  If the packet had come from the direction of N5, the situation
would have been reversed.  N5 would route the packet to N9 (being as close as
N8 and on the far side of N8.5).  N9 would route it back to N5 (since there
is not a node closer to N8.5, and N5 is the closest node on the opposite side
of N8.5 to N9).  N5 would process the node as it recieved it from a neighbor
which is closer to the destination that it is.
Clearly this will allow N8.5 to find its proper place whether it
is firewalled or not.
 
\subsection{Exiting the Network}
Exiting the network is pretty straight forward.  Ideally, a node sends a close
request to all its neighbors.  Of course, there will be some cases where a
close will not be sent (in the even of abrupt failures or loss of
connectivity).  When a node losses a connection, it may replace that
connection.  According to the protocol, every node keeps two outgoing
connections.  If the target of a nodes outgoing connections leaves, it must be
replaced.  Additionally, each node must attempt to be be connected to the two
nearest neighbors in the network (the firewalls frustrate that slightly, but
as we have seen it does not destroy routing).  Hence, if a node looses any
connection which is the closest on either side, that connection may be
replaced.  Lost local connections of nodes which are not outbound connections or
nearest neighbors do not need to be replaced\footnote{shortcut connections do
need to be replaced when lost, each node MUST have at least one shortcut
connection}
Since each node knows the addresses of all his neighbor's
neighbors, then the node can go about replacing the connection.  A second way
to verify that there are no hidden neighbors, is to send a connection request
to the node just on the other side of halfway between a node and its presumed
nearest neighbor.  Either the packet will be passed back, or it will reach a
neighbor in between, at which point a connection is made.  Both of these
approaches are used to verify that the ordering is proper after a node exits.

\section{Transports}
In a practical implementation of a protocol, it is not only important to have
good scaling properties, it is also neccesary to have low resource
requirements.  In the Brunet protocol, the node with the largest number of
unstructured connections will have $O(N)$ connections, and for file sharing
applications, each node will have $O(K)$ connections (where K is the number of
share contents).  The number of nodes in a network could easily grow to
hundreds of thousands or millions.  The number of files shared by a node might
be equally large.  At the same time, the number of TCP connections that most
operating systems support is considerably less than this.  As such, it is
obvious that if IP is to be used a transport, UDP must be the mechanism for
many, if not all, of the connections, however there are two problems with
using the UDP protocol for transport: it provides no recovery of lost packets
and it offers no flow control mechanism.

\subsection{Packet Loss}
Internet archetecture has worked remarkably well.  At the bottom, IP is a
datagram protocol with no guarantees on delivery of packets.  UDP is a thin
protocol that goes on top.  UDP packets can be up to 65535 bytes in size.
Those that are too large to be sent in one IP packet, are fragmented into many
IP packets.  If any one of those IP packets is lost, the entire UDP datagram
is lost.  Packet loss rates on the internet vary widely.  When a host is
overwhelmed by data, the packet loss rate will increase.  The packet loss rate
is not predictable or completely stable.  If the packet loss is very low, it
might be acceptable for the application.  However, for our networking
protocol, which is layered on top of the internet, the loss rate would be
critical.

Ordinarily, one can expect packet loss rates of around $1\%$ under
optimal conditions on the internet.  Since, our networking protocol requires
many hops between hosts on the internet, that rate will increase dramatically.
For instance, after 10 hops, that $1\%$ grows to $1-(1-0.01)^{10}\approx0.096$.
After 20 hops we get $1-(1-0.01)^{20}\approx0.182$.  Note that these figures are
based on an average of $1\%$ packet loss between any two hosts.  In fact, the
packet loss rate will be more properly modeled by a random variable which is
lognormally distributed (this comes from the multiplicative nature of combining
packet losses between hosts).

Note that some of the addresses in the Brunet system make use of sending packets
with a probability given by the address.  One might be tempted to use the fact
that UDP is unreliable to do some of the randomization for us.  The problem
with this approach is that when you send a packet, the probability that it makes
it to the destination is unknown and highly variable.  Any theory results which
depend on percolation theory will not apply if the percolation probability
itself is a random variable.  As such, it is probably best to handle the
randomization by selecting random edges, but then requiring acknowledgement that
the packet is recieved.

\subsection{Flow control}
An issue related to packet loss is flow control.  TCP offers a mechanism so that
one peer does not overwhelm the other peer with data.  This is flow control.
UDP offers no flow control, one peer may totally overwhelm the other with data.
When a peer is flooded with data, packet loss increases dramatically.  Hence,
the use of UDP without any mechanism for flow control, would likely increase the
packet loss rate.

The design of TCP is to allow the maximum throughput between two peers.  On the
other hand, the Brunet system is not designed for high throughput applications.
In fact, the exact opposite is true: Brunet attempts to provide a low bandwidth
solution for applications requiring decentralized queries and location
independent resource resolution.  As such, a very simple flow control mechanism
should be sufficient for Brunet.  In particular, Brunet packets range from 100
bytes to 1000 bytes or so, and internet latency is about 100ms.  If a Brunet
implementation allows n packets of size b to be transit before requiring
acknowledgement, and the round trip time is t, then the rate of the connection
will simply be: $R=\frac{nb}{t}$.  If the packet size is about 200 bytes, and
the round trip time is about 200 ms, then we see $R=n$ where R is measured in
1000 bytes per second.

So, the UDP transport described in the Brunet specification provides
acknowledgement of packets.  The specification does not say how an application
is to perform with regard to those packets.  By limiting the number of packets
a Brunet implementation will send before recieving acknowledgement, a peer can
limit the data rate of the packets sent.  By limiting the rate at which a node
will acknowledge packets, it can control the rate at which it recieves data.

The key to the UDP transport for Brunet is to offer mechanism to control packet
loss and to implement simple flow control while not requiring large resources.
An implementation should be able to handle millions of connections on the UDP
transport.

\section{Encrypting Datagrams Between Nodes}
The Brunet system uses randomly generated addresses for many of the addresses in
the system.  This offers some security benefits that do not exist in some other
protocols.  Rather than selecting addresses at random, addresses could be
derived from a cryptographic hash of some other data.  As such, the address
serves authenticates itself.

For instance, the entire network might use the Diffie-Hellman key agreement
protocol.  Each node could have a private key and a public key.  As an example,
consider a node with public key $g^a \ {\rm mod}\ p$, and private key $a$.
The address could be selected as the SHA1 hash of the public key.  Of course, in
order to make this a class 0 address, the last bit is replaced with '0'.
When two nodes communicate their keys to each other, they each can verify that
the key they have is the proper key for the address.
In this way,
when an address gives its public key, it is clear that the public key does
indeed belong to this node.  This can be done for any public key protocol.

\section{Content and Resource Addresses}
In order to resolve (or find nodes which have) resources, we need a way to
represent content as nodes in the network.  One way to do this, is that each
host which wants knows how to find content connects to the network with a class
1 address.  One way to do this is to take the URN (uniform resource name) for
the resource, and hash that name with the SHA1 hash.  Finally, replace the last
two bits with '01', and the result will be a class 1 address.  The node with the
address obtained by this procedure may then be queried for the particular URN.

\end{document}
